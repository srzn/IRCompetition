\documentclass{exam}
\usepackage{booktabs}
\usepackage{tasks}
\usepackage[T1]{fontenc}
\usepackage{graphicx}
\usepackage{tabularx}
\usepackage{nonfloat}
\usepackage{caption}
\usepackage{amsmath,amssymb}
\usepackage{lmodern}
\usepackage{textcomp}
\usepackage{ifpdf}
\usepackage[export]{adjustbox}
\usepackage{graphicx}
\usepackage{pdfpages}
\usepackage{titlesec}
\usepackage{etoolbox}
\usepackage{bigints}
\newcommand\printqtags{\def\qtag##1{\par\textbf{Tag: (##1)}}}
\newcommand\noprintqtags{\def\qtag##1{}}
\unframedsolutions
\printanswers
\printqtags
\AfterEndEnvironment{question}{\vspace{\stretch{1}}}

\begin{document}
\title{Project Proposal for CS 410}
\maketitle
\begin{questions}
\question
	\textbf{Name, NetId and Captain}
	\\\textit{*This is a one member team*} \\
	Name: Sai Srujan Gudibandi \\
	NetId: ssg7 \\
	Captain: Sai Srujan Gudibandi \\
\vspace{\stretch{10}}
\question
	\textbf{Competition selection}
	\\ I choose to participate in the Information Retrieval Competition
\vspace{\stretch{10}}

\question
	\textbf{Implementation and ideas for the project}
	\begin{tasks}(1)
		\task[i)] I plan to use a supervised learning approach owing to the fact that IR is more amenable to empirical tuning over a completely black-box approach of constructing an unsupervised mathematical model 
		\task[ii)] From the experience of constructing an IR system for MP2.4, I have realized that Okapi-BM25 performed best for the cranfield dataset. While constructing different rankers, I have cycled through various models like Pivoted Length Normalization, INL2 ranking, Zhai's BM25+, Jelinek-Mercer, and Dirichlet smoothing models. After extensive and exhaustive tuning on each and every model, Okapi-BM25 emerged victorious over the others. I, however, didn't have the chance to implement learning-to-rank methods by combining more than one ranking models to obtain a superior performance. I would start with implementing one of the learning-to-rank methodologies described in the optional module of the course to pitch it against individual models to compare their performances.  
		\task[iii)] In implementing learing-to-rank, I wish to choose a classification based learning to rank from among the different approaches. For this purpose, I'm currently exploring SVM$^{map}$. I may change my choice of LTRs and choose another model if I'm not satisfied with the performance of a previously chosen one. In addition to working on the datasets provided for the competition, I plan on using LETOR as a benchmark dataset for my model.
	\end{tasks} 
\vspace{\stretch{10}}
\question
	\textbf{Programming language option}
	\\ I wish to implement the majority of the project in Python. However, I may use C++ and/or R in the due course of the project if need be the details of which I will thoroughly document in the user guide.
\vspace{\stretch{10}}
\subsection*{References}
	Liu,``Learning to rank for Information Retrieval'', \textit{Foundations and Trends\textregistered \:in Information Retrieval}, Vol. 3 No. 3 (2009) pp. 225–331, 2009 \\ \\
	http://projects.yisongyue.com/svmmap/ \\ \\
	https://www.microsoft.com/en-us/research/publication/letor-benchmark-collection-research-learning-rank-information-retrieval/
\end{questions}

\end{document}